\section{Experimental Overview}

In the next few sections of the paper, we will give detailed explanations of the experimental tasks we consider. As a preview, the tasks we consider are:

\begin{figure*}[t]
	\centering
	\resizebox{!}{3.5in}{
	   \begin{tikzpicture}[->,>=stealth']
\node[state] (TWEET) 
{\begin{tabular}{l}
	\textbf{Tweet}\\
	\definevariableobject{User user;}\\
	\definevariableobject{Coordinate coordinates;}\\
	\definevariableobject{Place place;}\\
	\definevariableobject{Tweet quoted\_status;}\\
	\definevariableobject{Tweet retweeted\_status;}\\
	\definevariableobject{Entity entities;}\\
	\definevariableobject{ExtendedEntity extended\_entities;}\\
	\definevariableobject{List$\textless$MatchingRule$\textgreater$ matching\_rules;}\\
	\ \  \ldots \\
	\definevariable{String created\_at;}\\
	\definevariable{long id;}\\
	\definevariable{String text;}\\
	\definevariable{String source;}\\
	\definevariable{boolean truncated;}\\
	\definevariable{Map$<$String,Boolean$>$ scopes;}\\
	\definevariable{List$<$String$>$ withheld\_in\_countries;}\\
	\ldots  \ldots  \ldots \ldots  \ldots \\
	\textcolor{blue}{\textbf{\#tweet= 1,000,000}}\\
	\textcolor{blue}{\textbf{\#retweeted\_status= 598,517}}\\
	\textcolor{blue}{\textbf{\#quoted\_status= 177,537}}\\
	\end{tabular}};

% Next node: User
\node[state,       
node distance=7cm,    
%text width=3cm,       
right of=TWEET,        
yshift=2cm] (USER)   
{                    
	\begin{tabular}{l}  
	\textbf{User}\\
	\definevariableobject{List$<$Url$>$ description\_url;}\\
	\ \  \ldots \\
	\definevariable{long id;}\\
	\definevariable{String name;}\\
	\definevariable{String location;}\\
	\definevariable{String description;}\\
	\definevariable{String created\_at;}\\
	\ldots  \ldots  \ldots \ldots  \ldots \\
	\textcolor{blue}{\textbf{\# 1,000,000}}\\
	\end{tabular}
};

% Next node: Coordinates
\node[state, yshift=-5.5cm] (COORDINATE) 
{
	\begin{tabular}{l}   
	\textbf{Coordinate}\\
	\definevariable{String type;}\\
	\definevariable{double$[\ ]$ coordinates;}\\
	\ldots  \ldots  \ldots \ldots  \ldots \\
	\textcolor{blue}{\textbf{\# 1,586}}\\
	\end{tabular}
};
% Next node: MatchingRule
\node[state, yshift=-7.6cm](MatchingRule)
{                    
	\begin{tabular}{l}   
	\textbf{MatchingRule}\\
	\definevariable{long id;}\\
	\definevariable{String id\_str;}\\
	\definevariable{String tag;}\\
	% \ldots  \ldots  \ldots \ldots  \ldots \\
	%\textcolor{blue}{\textbf{\# 1,586}}\\
	\end{tabular}
};

% Next node: Coordinates
\node[state,
node distance=7.6cm,    
%text width=3cm,       
right of=TWEET,        
yshift=+-1.9cm] (ENTITY)    
{%                    
	\begin{tabular}{l}     
	\textbf{Entity}\\
	\definevariableobject{List$<$Hashtag$>$ hashtags;}\\
	\definevariableobject{List$<$Media$>$ medias;}\\
	\definevariableobject{List$<$Url$>$ urls;}\\
	\definevariableobject{List$<$UserMention$>$ user\_mentions;}\\
	\definevariableobject{List$<$Symbol$>$ symbols;}\\
	\definevariableobject{List$<$Poll$>$ polls;}\\
	\ldots  \ldots  \ldots \ldots  \ldots \\
	\textcolor{blue}{\textbf{\# 1,000,000}}\\
	\end{tabular}
};
% Next node: ExtendedEntity
\node[state,      
node distance=6.5cm,     
%text width=3cm,        
right of=TWEET,        
yshift=+-4.8cm] (EXTENDEDENTITY)    
{%                     
	\begin{tabular}{l}     
	\textbf{ExtendedEntity}\\
	\definevariableobject{List$<$Media$>$ medias;}\\
	\ldots  \ldots  \ldots \ldots  \ldots \\
	\textcolor{blue}{\textbf{\# 51,479}}\\
	\end{tabular}
};

% Next node: Place
\node[state,       
node distance=7cm,     
%text width=3cm,        
right of=TWEET,       
yshift=+-7.6cm] (PLACE)    
{%                    
	\begin{tabular}{l}     
	\textbf{Place}\\
	\definevariableobject{Bounding\_Box bounding\_box;}\\
	\ \  \ldots \\
	\definevariable{String name;}\\
	\definevariable{String country\_code;}\\
	\definevariable{String id;}\\
	\definevariable{String country;}\\
	\ldots  \ldots  \ldots \ldots  \ldots \\
	\textcolor{blue}{\textbf{\# 11,974}}\\
	
	\end{tabular}
};

% Next node: BoundingBoxCoordinate
\node[state,       
node distance=5.9cm,     
%text width=3cm,        
right of=TWEET,        
yshift=-10.7cm] (BoundingBoxCoordinate)    
{%                     
	\begin{tabular}{l}     
	\textbf{BoundingBox}\\	
	\definevariable{String type;}\\
	\definevariable{List$<$List$<$List$<$double$>>>$ coordinates;}\\	
	%\ldots  \ldots  \ldots \ldots  \ldots \\
	%\textcolor{blue}{\textbf{\# 11,974}}\\	
	\end{tabular}
};

private String type;
private List<List<List<Double>>> coordinates;

% Next node: Urls
\node[state,       
node distance=6.5cm,    
right of=USER,       
yshift=0.5cm] (URL)   
{
	\begin{tabular}{l}    
	\textbf{Url}\\
	\definevariable{List$<$Integer$>$ indices;}\\
	\definevariable{String display\_url;}\\
	\definevariable{String expanded\_url;}\\
	\definevariable{String url;}\\
	\ldots  \ldots  \ldots \ldots  \ldots \\
	\textcolor{blue}{\textbf{\# 417,178}}\\
	
	\end{tabular}
};

% Next node: Hashtag
\node[state,       
node distance=6.5cm,    
right of=USER,       
yshift=-2cm] (HASHTAG)   
{
	\begin{tabular}{l}    
	\textbf{Hashtag}\\
	\definevariable{List$<$Integer$>$ indices;}\\
	\definevariable{String text}\\	
	\ldots  \ldots  \ldots \ldots  \ldots \\
	\textcolor{blue}{\textbf{\# 398,720}}\\
	\end{tabular}	
};

% Next node: Hashtag
\node[state,       
node distance=6.5cm,    
right of=USER,       
yshift=-4.8cm] (USERMENTION)   
{
	\begin{tabular}{l}    
	\textbf{UserMention}\\
	\definevariable{long id;}\\
	\definevariable{String id\_str;}\\
	\definevariable{List$<$Integer$>$ indices;}\\
	\definevariable{String name;}\\
	\definevariable{String screen\_name;}\\	
	\ldots  \ldots  \ldots \ldots  \ldots \\
	\textcolor{blue}{\textbf{\# 1,063,216}}\\
	\end{tabular}	
};

% Next node: Symbol
\node[state,       
node distance=6.5cm,    
right of=USER,       
yshift=-7.6cm] (SYMBOL)   
{
	\begin{tabular}{l}    
	\textbf{Symbol}\\	
	\definevariable{List$<$Integer$>$ indices;}\\
	\definevariable{String text;}\\	
	\ldots  \ldots  \ldots \ldots  \ldots \\
	\textcolor{blue}{\textbf{\# 7,793}}\\
	\end{tabular}	
};

% Next node: Poll
\node[state,       
node distance=6.5cm,    
right of=USER,       
yshift=-9.6cm] (POLL)   
{
	\begin{tabular}{l}    
	\textbf{Poll}\\	
	\definevariableobject{List$<$Option$>$ options;}\\
	\definevariable{String end\_datetime;}\\	
	\definevariable{String duration\_minutes;}\\	
	%\ldots  \ldots  \ldots \ldots  \ldots \\
	%\textcolor{blue}{\textbf{\# 7,793}}\\
	\end{tabular}	
};
% Next node: Media
\node[state,       
node distance=7.3cm,    
right of=USER,       
yshift=-12.3cm] (MEDIA)   
{
	\begin{tabular}{l}    
	\textbf{Media}\\	
	\definevariableobject{MediaSize sizes;}\\
	\definevariableobject{Video video\_info;}\\
	\definevariableobject{AdditionalMedia additional\_media;}\\
	\ \  \ldots \\
	\definevariable{List$<$Integer$>$ indices;}\\
	\definevariable{String url;}\\
	\ldots  \ldots  \ldots \ldots  \ldots \\
	\textcolor{blue}{\textbf{\# 51,481}}\\
	\end{tabular}	
};

% Next node: Media
\node[state,       
node distance=5.8cm,    
right of=URL,       
yshift=0.7cm] (OPTION)   
{
	\begin{tabular}{l}    
	\textbf{Option}\\		
	\definevariable{int position;}\\
	\definevariable{String text;}\\
	%\ldots  \ldots  \ldots \ldots  \ldots \\
	%\textcolor{blue}{\textbf{\# 51,481}}\\
	\end{tabular}	
};

% Next node: MediaSize
\node[state,       
node distance=2cm,    
below of=OPTION,       
yshift=-0.1cm] (MEDIASIZA)   
{
	\begin{tabular}{l}    
	\textbf{MediaSize}\\		
	\definevariableobject{Size thumb;}\\
	\definevariableobject{Size large;}\\
	\definevariableobject{Size medium;}\\
	\definevariableobject{Size small;}\\
	\ldots  \ldots  \ldots \ldots  \ldots \\
	\textcolor{blue}{\textbf{\# 51,485}}\\
	\end{tabular}	
};

% Next node: Size
\node[state,       
node distance=6.3cm,    
right of=USERMENTION,       
yshift=0.3cm] (SIZE)   
{
	\begin{tabular}{l}    
	\textbf{Size}\\		
	\definevariable{int width;}\\
	\definevariable{int height;}\\
	\definevariable{String resize;}\\
	
	\ldots  \ldots  \ldots \ldots  \ldots \\
	\textcolor{blue}{\textbf{\#thumb= 51,485}}\\
	\textcolor{blue}{\textbf{\#large= 51,485}}\\
	\textcolor{blue}{\textbf{\#medium= 51,485}}\\
	\textcolor{blue}{\textbf{\#small= 51,485}}\\
	\end{tabular}	
};

% Next node: Video
\node[state,       
node distance=7.2cm,    
right of=SYMBOL,       
yshift=0.3cm] (VIDEO)   
{
	\begin{tabular}{l}    
	\textbf{Video}\\		
	\definevariableobject{List$<$Variant$>$ variants;}\\
	\definevariable{List$<$Integer$>$ aspect\_ratio;}\\
	\definevariable{int duration\_millis;}\\
	
	%\ldots  \ldots  \ldots \ldots  \ldots \\
	%\textcolor{blue}{\textbf{\#thumb= 51,485}}\\
	
	\end{tabular}	
};
% Next node: Variant
\node[state,       
node distance=6.6cm,    
right of=SYMBOL,       
yshift=-1.9cm] (Variant)   
{
	\begin{tabular}{l}    
	\textbf{Variant}\\		
	\definevariable{long bitrate;}\\
	\definevariable{String content\_type;}\\
	\definevariable{String url;}\\
	
	%\ldots  \ldots  \ldots \ldots  \ldots \\
	%\textcolor{blue}{\textbf{\#thumb= 51,485}}\\
	
	\end{tabular}	
};

% Next node: AdditionalMedia
\node[state,       
node distance=5.8cm,    
right of=MEDIA,       
yshift=0cm] (AdditionalMedia)   
{
	\begin{tabular}{l}    
	\textbf{AdditionalMedia}\\		
	\definevariable{String title;}\\
	\definevariable{String description;}\\
	\definevariable{boolean embeddable;}\\
	\definevariable{boolean monetizable;}\\	
	%\ldots  \ldots  \ldots \ldots  \ldots \\
	%\textcolor{blue}{\textbf{\#thumb= 51,485}}\\
	
	\end{tabular}	
};

\draw[connect] (3.3,2) -- (4.4,2) ;% TWEET -> User
\draw[connect] (3.3,-1) -- (4.5,-1) ;% TWEET -> User
\draw[connect] (3, -3.95) -- (3,-4.7) (3,-4.7) -- (4.5,-4.7) ;% TWEET -> Extended Entity
\draw[connect] (2.8, -3.95) -- (2.8,-7.5) (2.8,-7.5) -- (4.5,-7.5) ; % TWEEt -> Place
\draw[connect] (-2.7, -3.95) -- (-2.7,-7.5) (-2.7,-7.5) -- (-1.45,-7.5) ; % TWEEt -> Matching Rules
\draw[connect] (9.6,3) -- (11.2,3) ;% User -> Url
\draw[connect] (10,2)  -- (10,-0.18)  (10,2) -- (11.2,2) ;% Entity -> Url
\draw[connect] (7,-9.3)  -- (7,-10.05) ;% Place -> BondBox
\draw[connect] (10.65,-0.6) -- (11.2,-0.6) ;% Entity -> Hashtag
\draw[connect] (10.65,-2) -- (11.2,-2) ;% Entity -> User Mention
\draw[connect] (10.5,-3.65)  -- (10.5,-5.5)  (10.5,-5.5) -- (11.2,-5.5) ;% Entity -> User Mention
\draw[connect] (10.3,-3.65)  -- (10.3,-7.5)  (10.3,-7.5) -- (11.2,-7.5) ;% Entity -> Poll
\draw[connect] (10.1,-3.65)  -- (10.1,-9.5)  (10.1,-9.5) -- (11.2,-9.5) ;% Entity -> Media
\draw[connect]  (8.5,-5) --(9.9,-5) (9.9,-5)  -- (9.9,-10.2)  (9.9,-10.2) -- (11.2,-10.2) ;%  Extend Entity -> Media
\draw[connect](15.8,-7.3) -- (16.3,-7.3) (16.3,3)  -- (16.3,-7.3)  (16.3,3) -- (17.85,3) ;% Poll -> Option
\draw[connect] (16.5,1)  -- (16.5,-8.6)  (16.5,1) -- (17.95,1) ;% Media -> Media Size
\draw[connect] (16.7,-5.3)  -- (16.7,-8.6)  (16.7,-5.3) -- (18.05,-5.3) ;% Media -> Video
\draw[connect] (17.35,-10.5) -- (18.1,-10.5) ;% Media -> AdditionalMedia
\draw[connect] (18.5,-0.25)  -- (18.5,-0.75) ;% Media Size -> Size
\draw[connect] (19,-0.25)  -- (19,-0.75) ;% Media Size -> Size
\draw[connect] (19.5,-0.25)  -- (19.5,-0.75) ;% Media Size -> Size
\draw[connect] (20,-0.25)  -- (20,-0.75) ;% Media Size -> Size
\draw[connect] (20,-6.1)  -- (20,-6.7) ;% Media Size -> Size

\path %(TWEET) edge (USER)
(TWEET) edge (COORDINATE) ;
\end{tikzpicture}



	 }
	\caption{Object relationship and frequency of Tweet Objects (for one million tweets)}
	\label{fig:tweet_complex_object}
\end{figure*}

\begin{enumerate}
	\item A set of serialized objects stored externally on an HDD; the task is to read the objects into memory and deserialize them to their in-memory representation.
	\item A set of objects are stored in a large file (larger than the available RAM). The task is to perform an external sort of the file in order to perform a duplicate removal.
	\item A set of objects are partitioned across a number of machines in a network; the task is to send requests to the 	machines. Each machine answers the request by serializing the objects, then sending them over the network to the requesting machine.
	\item Finally, a set of sparse vectors are stored across various machines on a network. The task is to perform a tree aggregation where the vectors are aggregated over $log ( n )$
	hops.
\end{enumerate}

\subsection{Twitter Data Set}
For the various experiments, we use twitter data sets \cite{tweet_objects}, implemented using each of the ten different physical implementations. 

\subsection{Encoding sizes}
The ten different complex object implementations that we considered have very different encoding densities when the objects are serialized for storage or transmission across the network. The average, per-object sizes are given in Table %\ref{tbl:object_size}. 
%\begin{table}
%	\centering
%	\caption{Frequency of some Tweet Objects  (for 1 million tweets) }
%	\label{tbl:object_size}
%	\begin{adjustbox}{width=\columnwidth,center}	
		
%		\begin{tabular}{|c|c|c|} \hline
%			Object Name &Parent Object &Frequency\\ \hline
%			tweet  & root object& 1,000,000 \\ \hline
%			users & tweet & 1,000,000 \\ \hline
%			coordinates  &tweet& 1586 \\ \hline
%			place & tweet & 11974 \\ \hline
%			quoted status  & tweet & 177537 \\ \hline
%			retweeted status  & tweet & 598517 \\ \hline
%			entities  & tweet & 1,000,000 \\ \hline
%			extended entities  & tweet & 51479 \\ \hline
%			hashtags  & entities & 398720 \\ \hline
%			media  & entities & 51481 \\ \hline
%			urls  & entities & 417176 \\ \hline
%			user mentions  & entities & 1063216 \\ \hline
%			symbols  & entities & 7793 \\ \hline
%			sizes  & media & 7793 \\ \hline
%			media sizes  & sizes & 51485 \\ \hline
%			thumb  & media sizes & 51485 \\ \hline
%			large  & media sizes & 51485 \\ \hline
%			medium  & media sizes & 51485 \\ \hline
%			small  & media sizes & 51485 \\ \hline
%			\hline\end{tabular}
%	\end{adjustbox}
%\end{table}

\begin{table}
	\centering
	\caption{tweet complexity }
	\label{tbl:object_size}
	\begin{adjustbox}{width=\columnwidth,center}	
		
		\begin{tabular}{|c|c|} \hline
			Tweet type & Frequency\\ \hline
			Simple tweets(retweet \& quote are null ) & 332,901\\ \hline
			Retweets & 489,562\\ \hline
			Quote & 68,582\\ \hline
			Retweet \& Quote & 108,955\\ \hline
			Total & 1,000,000 \\ \hline
			
			\hline\end{tabular}
	\end{adjustbox}
\end{table}


\begin{table}
	\centering
	\caption{Object size for 1 million tweets and Lines of code for serialize/de-serialize }
	\label{tbl:object_size}
	\begin{adjustbox}{width=\columnwidth,center}	
		
		\begin{tabular}{|l|c|c|c|} \hline
		 \multirow{2}{*}{\textbf{Method}} & \multirow{2}{*}{\textbf{File size(Gig)}} & \textbf{Serialize} &  \textbf{De-Serialize}\\ 
		 &&\textit{Lines of code}& \textit{Lines of code} \\ \hline
			\texttt{Java Default}  & 4.6 & 4  & 4 \\ \hline	
			\texttt{Java Json+Gzip}  & 1.4  & 2 & 4 \\ \hline	
			\texttt{Java Bson}  & 4.9 & 50  & 120 \\ \hline	
			\multirow{2}{*}{\texttt{Java ProtoBuf}}  & \multirow{2}{*}{\texttt{1.9}} & 200 & 1 \\ 
										\cline{3-4}
										&  & \multicolumn{2}{c|}{\textcolor{red}{
												\texttt{with 20 extra files}}}   \\ \hline	
			\texttt{Java Kyro}  & 1.9  & 40 & 40 \\ \hline	
			\texttt{Java HandCoded ByteBuf$.$}  & 2.3  & 150 & 150 \\ \hline	
		    \multirow{2}{*}{\texttt{Java FaltBuffers}}  & \multirow{2}{*}{\texttt{2.9}} & 250 & 1 \\ 
		    				\cline{3-4}
		    				&&\multicolumn{2}{c|}{\textcolor{red}{
		    					\texttt{with 42 extra files}}}   
	    					\\ \hline
		    	
			\texttt{C++ HandCoded}  & 2.1 & 70  & 100 \\ \hline	
			\texttt{C++ InPlace}  & 3.2  & 80 & 1 \\ \hline	
			\texttt{C++ Boost}  & 2.2 & 1 & 2 \\ \hline				
			
			\multirow{2}{*}{\texttt{C++ ProtoBuf}}  & \multirow{2}{*}{\texttt{1.9}} & 200 & 1 \\ 
			\cline{3-4}
			&&\multicolumn{2}{c|}{\textcolor{red}{
					\texttt{with 20 extra files}}}   
			\\ \hline
			
			\texttt{C++ Bson}  & 4.6 & 40 & 100\\ \hline
			\multirow{2}{*}{\texttt{C++ FaltBuffers}}  & \multirow{2}{*}{\texttt{2.9}} & 250  & 1 \\ 
			\cline{3-4}
			&&\multicolumn{2}{c|}{\textcolor{red}{
					\texttt{with 42 extra files}}}   
			\\ \hline
				
			\texttt{Rust Serde Json}  & 4.8 & 1 & 1 \\ \hline
			\texttt{Rust Serde Bincode}  & 2.4 & 1 & 1\\ \hline			
			\texttt{Rust Serde MessagePack}  & 1.9 & 1 & 1 \\ \hline			
			\texttt{Rust Serde Bson}  & 4.5 & 1 & 1 \\ \hline			
			\texttt{Rust Serde FlexBuffers}  & 4.3 & 1 & 1 \\ 						
			\hline\end{tabular}
	\end{adjustbox}
\end{table}



\subsection{Experimental Details}
We run our experiments on Google Cloud costume instances which have 4 vCPU cores, 32 GB RAM and 3000 GB standard persistent disk(Sustained random IOPS limit: read=2,250 and write=4,500) running with Ubuntu Ubuntu 18.04.4 LTS. Before running each experiment task, we "warmed up" the Java Garbage Collector (GC) by creating a large number of objects. We do not include this warm-up-time in our performance time calculations.

We used two Java GC flags $-XX:-UseGCOverheadLimit$ and $-XX:+UseConcMarkSweepGC$. The first flag is used to avoid OutOfMemoryError exceptions while using the complete RAM size for data processing and the second flag is for running concurrent garbage collection.

We run all of our experiments 3 times and observed that the results have low variance. In this paper we present the average of those runs. Before running each experiment, we deleted th OS cache using the Linux command: $echo 3 > /proc/sys/vm/drop\_caches$.

Our Java implementation is written using Java 8 with the Oracle JDK version "1.8.0\_241" and for our C++ implementation we use the C++11, compiled using clang++ (version 6.0.0).

The source codes of our implementation and a brief description of technical details can be found on the Github Repository \footnote{The source code of our Implementation is available at \url{https://github.com/fathollahzadeh/serialization}} .
