\documentclass[10pt,journal,compsoc]{IEEEtran}
\usepackage{graphicx}
\usepackage{balance}  
\usepackage[utf8x]{inputenc}
\usepackage[colorinlistoftodos]{todonotes}
\usepackage{adjustbox}
%\usepackage{hyperref}
\usepackage{microtype}
\usepackage{pgfplots} 
\usepackage{multirow}
\usepackage{multicol}
\usepackage{algorithm}
\usepackage{url}
\usepackage{wrapfig}
\usepackage{lipsum}
\usepackage{tikz}
\usepackage{pgfplots}
\usepackage{textcomp}
\usetikzlibrary{shapes,arrows,fit,chains}
\usepackage{algorithm}
\usepackage{algpseudocode}
\usepackage{pgfplots, pgfplotstable}
\usepackage{tikz}
\usepackage{listings}
\usepackage{xcolor}
\usetikzlibrary{patterns}
\usetikzlibrary{backgrounds}
\usepackage{booktabs}
\usepackage{verbatim}
\usepackage{fancyvrb}
\usepackage{verbatimbox,lipsum}
\usepackage{listings}
\usepgfplotslibrary{groupplots}

%\pgfplotsset{compat=1.9}

% \usepackage{mathptmx}
% \usepackage{amsmath}
% \usepackage{amssymb}
% \usepackage{amsthm}


\newcommand{\definevariableobject}[1] {\textcolor{black}{ 
		\ttfamily\bfseries\textcolor{red}{\ #1}}
} 
\newcommand{\definevariable}[1] {\textcolor{black}{ 
		\ttfamily\bfseries\ #1}
} 
\definecolor{copper}{cmyk}{0,0.9,0.9,0.2}
\colorlet{lightgray}{black!25}
\colorlet{darkgray}{black!75}

\usepackage{xcolor}
\definecolor{tug}{RGB}{247,1,70}
\definecolor{tugb}{RGB}{120,137,251}

% \definecolor{color1}{RGB}{100,149,237} % corn flower blue
% \definecolor{color2}{RGB}{153,153,153} % light gray
% \definecolor{color3}{RGB}{0,0,0} % black
% \definecolor{color4}{RGB}{255,165,0} % orange
% \definecolor{color5}{RGB}{255,69,0} % orange red
% \definecolor{color6}{RGB}{77,77,77} % dark gray
% \definecolor{color7}{RGB}{31,119,180}
% \definecolor{color8}{RGB}{7,77,125}

\definecolor{color1}{RGB}{100,149,237} % corn flower blue
\definecolor{color2}{RGB}{153,153,153} % light gray
\definecolor{color3}{RGB}{0,0,0} % black
\definecolor{color4}{RGB}{255,165,0} % orange
\definecolor{color5}{RGB}{255,69,0} % orange red
\definecolor{color6}{RGB}{77,77,77} % dark gray
\definecolor{color7}{RGB}{31,119,180}
\definecolor{color8}{RGB}{7,77,125}

% Enable this two commands when you want to extract diagrams in extra files, then run "make"
\usetikzlibrary{external}
\tikzexternalize[prefix=plots/] %  activate

\newcommand{\showvaluewrite}[8]{
  \ifthenelse{
    \equal{#1}{#2} \OR
    \equal{#1}{#3} \OR
    \equal{#1}{#4} \OR
    \equal{#1}{#5} \OR
    \equal{#1}{#6} \OR
    \equal{#1}{#7} \OR
    \equal{#1}{#8} 
  }{\pgfkeys{/tikz/coordinate}}{}
}

\newcommand{\showvalue}[9]{
  \ifthenelse{
    \equal{#1}{#2} \OR
    \equal{#1}{#3} \OR
    \equal{#1}{#4} \OR
    \equal{#1}{#5} \OR
    \equal{#1}{#6} \OR
    \equal{#1}{#7} \OR
    \equal{#1}{#8} \OR
    \equal{#1}{#9} 
  }{\pgfkeys{/tikz/coordinate}}{}
}


% *** CITATION PACKAGES ***
%
\ifCLASSOPTIONcompsoc
  % IEEE Computer Society needs nocompress option
  % requires cite.sty v4.0 or later (November 2003)
  \usepackage[nocompress]{cite}
\else
  % normal IEEE
  \usepackage{cite}
\fi

% *** GRAPHICS RELATED PACKAGES ***
%
\ifCLASSINFOpdf
  % \usepackage[pdftex]{graphicx}
  % declare the path(s) where your graphic files are
  % \graphicspath{{../pdf/}{../jpeg/}}
  % and their extensions so you won't have to specify these with
  % every instance of \includegraphics
  % \DeclareGraphicsExtensions{.pdf,.jpeg,.png}
\else
  % or other class option (dvipsone, dvipdf, if not using dvips). graphicx
  % will default to the driver specified in the system graphics.cfg if no
  % driver is specified.
  % \usepackage[dvips]{graphicx}
  % declare the path(s) where your graphic files are
  % \graphicspath{{../eps/}}
  % and their extensions so you won't have to specify these with
  % every instance of \includegraphics
  % \DeclareGraphicsExtensions{.eps}
\fi

\makeatletter
\def\endthebibliography{%
	\def\@noitemerr{\@latex@warning{Empty `thebibliography' environment}}%
	\endlist
}
\makeatother

%\theoremstyle{definition}
%\newtheorem{findings}{Findings}


\hyphenation{op-tical net-works semi-conduc-tor}

\begin{document}

\title{Understanding and Benchmarking the Impact of Complex Object Implementations for Big Data Systems}

\author{Saeed Fathollahzadeh, Kia Teymourian, Chris Jermaine 
	\IEEEcompsocitemizethanks{
		\IEEEcompsocthanksitem 
		Saeed Fathollahzadeh is with the Department of Computer Science, Graz University of Technology, Graz, Austria.\protect\\ E-mail: s.fathollahzadeh@student.tugraz.at
		\IEEEcompsocthanksitem Kia Teymourian is with the Department of Computer Science, The University of Texas at Austin, TX, USA. \protect\\ E-mail: kiat@bu.edu
		\IEEEcompsocthanksitem Chris Jermaine is the Chair of Department of Computer Science, Rice University, Houston, TX, USA. \protect\\ E-mail: cmj4@rice.edu}	
\thanks{Manuscript received April 19, 2005; revised August 26, 2015.}}


% The paper headers
\markboth{Fathollahzadeh, et al: Understanding and Benchmarking the Impact of Complex Object Implementations for Big Data Systems}%
{Shell \MakeLowercase{\textit{et al.}}: Bare Demo of IEEEtran.cls for Computer Society Journals}

\IEEEtitleabstractindextext{%
\begin{abstract}

\end{abstract}

\begin{IEEEkeywords}
Computer Society, IEEE, IEEEtran, journal, \LaTeX, paper, template.
\end{IEEEkeywords}}

\tikzsetnextfilename{Exp1_write_single}
    \begin{figure*}[h]
        \centering
        \makeatletter
\newcommand\resetstackedplots{
	\makeatletter
	\pgfplots@stacked@isfirstplottrue
	\makeatother
	\addplot [forget plot,draw=none] coordinates{({HandcodedCPP},0.1) ({inPlaceCPP},0.1) ({ProtoBufCPP},0.1) ({FlatBufCPP},0.1)  ({BoostBinaryCPP},0.1) ({BoostCPP},0.1)  ({BsonCPP},0.1) 
	({MessagePackRust},0.1) ({BincodeRust},0.1)  ({JsonRust},0.1) ({FlexBufRust},0.1) ({BsonRust},0.1) 
	({ByteBufferJava},0.1) ({KryoJava},0.1) ({ProtoBufJava},0.1) ({FlatBufJava},0.1) ({DefaultJava},0.1) ({BsonJava},0.1)  ({JsonJava},0.1) ({Json+GzipJava},0.1)};
}
\makeatother


\begin{tikzpicture}
	\newcommand{\myaddplot}[8]{
	   \addplot[white,xshift=#2,draw=#5,line width=0.15pt, fill=#3, discard if single={#1}{#4}{#8}, postaction={pattern=#6,pattern color=#5}] 
	   table[ y=time, col sep=comma, x=baseline] {results/Exp1_write_single.dat};
	   \label{#4#1}
   };  

   \newcommand{\myaddplotnan}[9]{
	   \addplot[xshift=#2,draw=#9,line width=0.15pt, fill=#3, discard if single={#1}{#4}{#8}, %postaction={pattern=#6,pattern color=#5},
	   visualization depends on={value \thisrow{baseline} \as \xbaseline},
	   every node near coord/.append style={		  
		   check for zeronan/.code={
			   \pgfkeys{/pgf/fpu=true}
			   \showvalue{\xbaseline}{HandcodedCPP}{inPlaceCPP}{FlatBufCPP}{}{MessagePackRust}{BincodeRust}{JsonRust}{}{}
			   \pgfkeys{/pgf/fpu=false}
		   },
		   check for zeronan,
	   }  
	   ] 
	   table[ y=time, col sep=comma, x=baseline] {results/Exp1_write_single.dat};	  
	   \label{#4#1}
   };  
   
   \newcommand{\myaddplots}[6]{	   
	\addplot[xshift=#2,draw=#5,line width=0.15pt, fill=#3, discard if single={#1}{#4}{#6},
		forget plot,
		nodes near coords custom=1,
		nodes near coords={\pgfmathprintnumber[precision=1]{\pgfplotspointmeta}},
		every node near coord/.append style={
			black,			
			xshift=-10pt,
			anchor=west
		}
	] 
	table[ y=time, col sep=comma, x=baseline] {results/Exp1_write_single.dat};	
   };   
   
   \newcommand{\addDiagramExpOne}[9]{	
	\nextgroupplot[
		width=#4,		
		ytick=#5,
		yticklabels =#5,
		y axis line style={opacity=#6},
		#7,
		axis y line*=#8,
		%xlabel=#9,				
		] 
		\resetstackedplots	
		\myaddplot{True}{-6}{black}{IO}{black}{none}{I/O Time(taskset True)}{#1};
  		\myaddplotnan{True}{-6}{white}{CPU}{black}{north east lines}{CPU Time(taskset True)}{#1}{black};
		\myaddplots{True}{-6}{white}{Total}{none}{#1};
	    \resetstackedplots	 
		\myaddplot{False}{6}{gray!135}{IO}{gray!135}{none}{I/O Time(taskset False)}{#1};  
	    \myaddplotnan{False}{6}{lightgray!85}{CPU}{lightgray!85}{north west lines}{CPU Time(taskset False)}{#1}{black};
		\myaddplots{False}{6}{white}{Total}{none}{#1};
   };     

  \pgfplotsset{
	   discard if single/.style n args={3}{
		   x filter/.code={
			   \edef\tempa{\thisrow{taskset}}
			   \edef\tempb{#1}
			   \ifx\tempa\tempb
					   \edef\tempe{\thisrow{execution}}
					   \edef\tempf{#2}
					   \ifx\tempe\tempf	
					   	%%%%%%%%%%%%%%%%%%%
						   \edef\tempg{\thisrow{language}}
						   \edef\temph{#3}
						   \ifx\tempg\temph	
						   		%%%%%%%%%%%%%%%%%
								    \edef\tempi{\thisrow{platform}}
								    \edef\tempj{Single}
								    \ifx\tempi\tempj												   
								    \else
								    \def\pgfmathresult{inf}
								    \fi      
								%%%%%%%%%%%%%%%%%
						   \else
						   \def\pgfmathresult{inf}
						   \fi      
						%%%%%%%%%%%%%%%%%%%	
					   \else
					   \def\pgfmathresult{inf}
					   \fi      
			   \else
			   \def\pgfmathresult{inf}
			   \fi			
		   }
	   },
	   nodes near coords custom/.style={
		large value/.style={                    
			rotate=90,
			anchor=east,			
		},
		small value/.style={
			rotate=90,
			anchor=east,
		},
		every node near coord/.style={
		  font=\scriptsize,
		  inner sep=0.5mm,
		  /pgf/number format/1000 sep={},
		  check for zero/.code={%
			\pgfkeys{/pgf/fpu=true}
			\pgfmathparse{\pgfplotspointmeta-1}
			\pgfmathfloatifflags{\pgfmathresult}{0}{
				\pgfkeys{/tikz/coordinate}
			}{
				\begingroup                      
					\pgfkeys{/pgf/fpu}%
					\pgfmathparse{\pgfplotspointmeta<#1}%
					\global\let\result=\pgfmathresult
				\endgroup
				\pgfmathfloatcreate{1}{1.0}{0}%
				\let\ONE=\pgfmathresult
				\ifx\result\ONE
					% AH : our condition 'y < #1' is met.
					\pgfkeysalso{/pgfplots/small value}%
				\else
					% ok, proceed as usual.
					\pgfkeysalso{/pgfplots/large value}%
				\fi
			}
			\pgfkeys{/pgf/fpu=false}
		  },
		  check for zero,
		},
	},   
   };
   %%%%%%%%%%%%%%%%%%%%%%%%%%%%%%%%%%%%%%%%%%%%%%%%
   \begin{groupplot}[
		group style={
			group size=3 by 1,
			xlabels at=edge bottom,
			ylabels at=edge left,
			horizontal sep=0pt,
			vertical sep=0pt,
			/pgf/bar width=12pt
		},			
		axis line style={gray},
		ybar stacked,        
		ymode=log,
		ymin=0.2,
		ymax=100,
		scaled y ticks=false,		  
		enlarge y limits={0.15,upper},		
		ylabel={Execution Time [min]},		   
		ytick align=inside,
		xtick align=outside,			
		xtick pos=left,
		ytick pos=left,
		yticklabel style = {font=\footnotesize},
		ylabel style = { yshift=-12pt},
		xticklabel style = {font=\footnotesize},
		x label style={yshift=-20pt},
		xtick=data,
		height=0.7\columnwidth,  
		symbolic x coords={ HandcodedCPP, inPlaceCPP, ProtoBufCPP, FlatBufCPP, BoostBinaryCPP, BoostCPP, BsonCPP, MessagePackRust,BincodeRust, JsonRust, FlexBufRust, BsonRust,ByteBufferJava,KryoJava,ProtoBufJava,FlatBufJava,DefaultJava,BsonJava,JsonJava, Json+GzipJava},
		%
		xticklabels={HandCoded, , ProtoBuf, , BoostBinary, , BSON, ,Bincode, , FlexBuf, ,ByteBuffer,,ProtoBuf,,Default,,JSON, },  
		%
		extra x ticks={inPlaceCPP,FlatBufCPP,BoostCPP,MessagePackRust,JsonRust,BsonRust,KryoJava,FlatBufJava,BsonJava,Json+GzipJava},
		extra x tick labels={InPlace,FlatBuf,BoostText,MessagePack,JSON,BSON,Kryo,FlatBuf,BSON,JSON+Gzip},
		%
		every extra x tick/.style={major tick length=18pt, xtick align=outside},
		point meta=rawy,    		
		nodes near coords={\pgfmathprintnumber[fixed,assume math mode=true,precision=1]{\pgfplotspointmeta}},
		nodes near coords custom={1}, 
		legend image code/.code={\draw [#1] (-0.2cm,-0.1cm) rectangle (0.20cm,0.20cm); },
		xticklabel style={name=T\ticknum}			
]
\addDiagramExpOne{CPP}{tugb}{color7}{0.48\textwidth}{{10,100}}{1}{{enlarge x limits=0.12,}}{left}{C++};		
\node [draw=none,inner sep=0, font=\footnotesize, anchor=west](leg1) at (rel axis cs: 0.1,0.77) {\shortstack[l]{
		\ref{CPUTrue} CPU Time (taskset True) \\ 
		\ref{IOTrue} I/O Time (taskset True) \\
		\ref{CPUFalse} CPU Time (taskset False) \\ 
		\ref{IOFalse} I/O Time (taskset False) 				
}};
\addDiagramExpOne{Rust}{tugb}{color7}{0.40\textwidth}{\empty}{0}{{enlarge x limits=0.24,}}{left}{Rust};
\addDiagramExpOne{Java}{tugb}{color7}{0.55\textwidth}{\empty}{1}{{enlarge x limits=0.12,}}{right}{Java};
		
\end{groupplot}

\begin{scope}[decoration=brace]
	\pgfdecorationsegmentamplitude=5pt
	\draw[decorate] ($(T2.south east)+(25pt,0)$) -- ($(T0.south west)+(-25pt,0)$) node[midway,below=\pgfdecorationsegmentamplitude] {\footnotesize{C++}};
	\draw[decorate] ($(T4.south east)+(60pt,0)$) -- ($(T3.south west)+(5pt,0)$) node[midway,below=\pgfdecorationsegmentamplitude] {\footnotesize{Rust}};
	\draw[decorate] ($(T8.south east)+(58pt,0)$) -- ($(T5.south west)+(43pt,0)$) node[midway,below=\pgfdecorationsegmentamplitude] {\footnotesize{Java}};	
  \end{scope}
\end{tikzpicture}
        \caption{Experiment1: Write times for 5M Tweet Objects (Single)}
\end{figure*}

\tikzsetnextfilename{Exp1_write_parallel}
    \begin{figure*}[h]
        \centering
        \makeatletter
\newcommand\resetstackedplots{
	\makeatletter
	\pgfplots@stacked@isfirstplottrue
	\makeatother
	\addplot [forget plot,draw=none] coordinates{({HandcodedCPP},0.01) ({inPlaceCPP},0.01) ({FlatBufCPP},0.01) ({ProtoBufCPP},0.01) 
	({BoostCPP},0.01) ({BsonCPP},0.01) ({MessagePackRust},0.01) ({BincodeRust},0.01) ({JsonRust},0.01) ({FlexBufRust},0.01) ({BsonRust},0.01) ({Json+GzipJava},0.01) 
	({JsonJava},0.01) ({BsonJava},0.01) ({DefaultJava},0.01) ({ProtoBufJava},0.01) ({ByteBufferJava},0.01) ({FlatBufJava},0.01) ({KryoJava},0.01)};
}
\makeatother

\begin{tikzpicture}
	\newcommand{\myaddplot}[7]{
	   \addplot[xshift=#1,draw=#4,line width=0.15pt, fill=#2, discard if single={#3}{#7}, postaction={pattern=#5,pattern color=#4}] 
	   table[ y=time, col sep=comma, x=baseline] {results/Exp1_write.dat};
	   \label{#3#7}
   };  

   \newcommand{\myaddplotnan}[8]{
	   \addplot[xshift=#1,draw=#8,line width=0.15pt, fill=#2, discard if single={#3}{#7},
	   visualization depends on={value \thisrow{baseline} \as \xbaseline},
	   every node near coord/.append style={		  
		   check for zeronan/.code={
			   \pgfkeys{/pgf/fpu=true}
			   \showvalue{\xbaseline}{HandcodedCPP}{inPlaceCPP}{FlatBufCPP}{ProtoBufCPP}{BoostCPP}{MessagePackRust}{JsonRust}{FlexBufRust}{BsonRust};
			   \showvalue{\xbaseline}{BincodeRust}{BsonRust}{BsonJava}{DefaultJava}{}{}{}{}{}
			   \pgfkeys{/pgf/fpu=false}
		   },
		   check for zeronan,
	   }  
	   ] 
	   table[ y=time, col sep=comma, x=baseline] {results/Exp1_write.dat};	  
	   \label{#3#7}
   };  
   
   \newcommand{\myaddplots}[5]{	   
	\addplot[xshift=#1,draw=#4,line width=0.15pt, fill=#2, discard if single={#3}{#5},
		forget plot,
		nodes near coords custom=1,
		nodes near coords={\pgfmathprintnumber[precision=1]{\pgfplotspointmeta}},
		every node near coord/.append style={
			black,			
			xshift=-7pt,
			anchor=west
		}
	] 
	table[ y=time, col sep=comma, x=baseline] {results/Exp1_write.dat};	
   };   
   
   \newcommand{\addDiagramExpOne}[7]{	
	\nextgroupplot[
		width=#4,		
		yticklabels=#5,
		ytick=#6,
		#7
		]
		\resetstackedplots	
		\myaddplot{0}{#2}{IO}{#2}{none}{I/O Time(taskset True)}{#1};
  		\myaddplotnan{0}{#2!40}{CPU}{#2!40}{north east lines}{CPU Time(taskset True)}{#1}{#2};
		\myaddplots{0}{white}{Total}{none}{#1};
   };     

  \pgfplotsset{
	   discard if single/.style n args={2}{
		   x filter/.code={
		   \edef\tempe{\thisrow{execution}}
		   \edef\tempf{#1}
		   \ifx\tempe\tempf	
			   \edef\tempg{\thisrow{language}}
			   \edef\temph{#2}
			   \ifx\tempg\temph	
				    \edef\tempi{\thisrow{platform}}
				    \edef\tempj{Parallel}
				    \ifx\tempi\tempj												   
				    \else
				    \def\pgfmathresult{inf}
				    \fi      
			   \else
			   \def\pgfmathresult{inf}
			   \fi      
		   \else
		   \def\pgfmathresult{inf}
		   \fi      
	   }
	   },
	   nodes near coords custom/.style={
		large value/.style={                    
			rotate=90,
			anchor=east,			
		},
		small value/.style={
			rotate=90,
			anchor=east,
		},
		every node near coord/.style={
		  font=\scriptsize,
		  inner sep=0.5mm,
		  /pgf/number format/1000 sep={},
		  check for zero/.code={%
			\pgfkeys{/pgf/fpu=true}
			\pgfmathparse{\pgfplotspointmeta-0.1}
			\pgfmathfloatifflags{\pgfmathresult}{0}{
				\pgfkeys{/tikz/coordinate}
			}{
				\begingroup                      
					\pgfkeys{/pgf/fpu}%
					\pgfmathparse{\pgfplotspointmeta<#1}%
					\global\let\result=\pgfmathresult
				\endgroup
				\pgfmathfloatcreate{1}{0.1}{0}%
				\let\ONE=\pgfmathresult
				\ifx\result\ONE
					% AH : our condition 'y < #1' is met.
					\pgfkeysalso{/pgfplots/small value}%
				\else
					% ok, proceed as usual.
					\pgfkeysalso{/pgfplots/large value}%
				\fi
			}
			\pgfkeys{/pgf/fpu=false}
		  },
		  check for zero,
		},
	},   
   };
   %%%%%%%%%%%%%%%%%%%%%%%%%%%%%%%%%%%%%%%%%%%%%%%%
   \begin{groupplot}[
		group style={
			group size=3 by 1,
			xlabels at=edge bottom,
			ylabels at=edge left,
			horizontal sep=4pt,
			vertical sep=0pt,
			/pgf/bar width=8pt
		},			
		axis line style={gray},
		ybar stacked,        
		ymode=log,
		ymin=0.3,
		x tick label style={/pgf/number format/1000 sep=},
		ymax=400,		
		scaled y ticks=false,		  
		enlarge y limits={0.3,upper},		
		ylabel={Execution Time[s]},		   
		ytick align=inside,
		xtick align=outside,			
		xtick pos=left,
		ytick pos=left,
		yticklabel style = {font=\scriptsize},
		ylabel style = {font=\scriptsize, yshift=-12pt},
		xticklabel style = {font=\scriptsize, rotate=50, anchor=east, xshift=1pt, yshift=-1pt},
		x label style={yshift=-20pt},
		xtick=data,
		height=0.7\columnwidth,  
		symbolic x coords={ HandcodedCPP, inPlaceCPP, FlatBufCPP, ProtoBufCPP, BoostCPP, BsonCPP, MessagePackRust, BincodeRust, JsonRust, FlexBufRust, BsonRust, Json+GzipJava, JsonJava, BsonJava, DefaultJava, ProtoBufJava, ByteBufferJava, FlatBufJava, KryoJava},		   
		xticklabels={Handcoded, inPlace, FlatBuf, ProtoBuf, Boost, Bson, MessagePack, Bincode, Json, FlexBuf, Bson, Json+Gzip, Json, Bson, Default, ProtoBuf, ByteBuffer, FlatBuf, Kryo},  		   
		point meta=rawy,            
		nodes near coords={%
    		\pgfmathprintnumberto[fixed,assume math mode=true,precision=1]{\pgfplotspointmeta}{\myval}%
    		\pgfmathparse{\myval<=0.1?:\myval}\pgfmathresult%
		},
		nodes near coords custom={1}, 
		legend image code/.code={\draw [#1] (0cm,-0.1cm) rectangle (0.20cm,0.20cm); },
		xticklabel style={name=T\ticknum}			
]
\addDiagramExpOne{CPP}{tug}{color7}{0.21\textwidth}{{10,100,1000,10000}}{{10,100,1000}}{{enlarge x limits=0.15,}};	
\node [draw=none,inner sep=2, fill=lightgray, text width=0.12\textwidth,align=center,font=\scriptsize, anchor=west](leg1) at (rel axis cs: 0.00,0.965) {C++};
\node [draw=none,inner sep=0, font=\scriptsize, anchor=west](leg1) at (rel axis cs: 0.2,0.83) {\shortstack[l]{
		\ref{CPUCPP} CPU \\ \\ 
		\ref{IOCPP} IO 
}};

\addDiagramExpOne{Rust}{tugb}{color7}{0.19\textwidth}{}{}{{enlarge x limits=0.18,}};
\node [draw=none,inner sep=2, fill=lightgray, text width=0.1\textwidth,align=center,font=\scriptsize, anchor=west](leg1) at (rel axis cs: 0.00,0.965) {Rust};
\node [draw=none,inner sep=0, font=\scriptsize, anchor=west](leg1) at (rel axis cs: 0.2,0.83) {\shortstack[l]{
		\ref{CPURust} CPU \\ \\ 
		\ref{IORust} IO 
}};

\addDiagramExpOne{Java}{color4}{color7}{0.26\textwidth}{}{}{{enlarge x limits=0.14,}};
\node [draw=none,inner sep=2, fill=lightgray, text width=0.17\textwidth,align=center,font=\scriptsize, anchor=west](leg1) at (rel axis cs: 0.00,0.965) {Java};
\node [draw=none,inner sep=0, font=\scriptsize, anchor=west](leg1) at (rel axis cs: 0.7,0.83) {\shortstack[l]{
		\ref{CPUJava} CPU \\ \\ 
		\ref{IOJava} IO 
}};
\end{groupplot}
   
\end{tikzpicture}
        \caption{Experiment1: Write times for 5M Tweet Objects (parallel)}
\end{figure*}



% \tikzsetnextfilename{Exp2_read_memory}
%     \begin{figure}[h]
%         \centering
%         
\begin{tikzpicture}
	\newcommand{\myaddplot}[7]{
	   \addplot[xshift=#2,draw=#5,line width=0.15pt, fill=#3, discard if single={#1}{#4}, postaction={pattern=#6,pattern color=#5}] 
	   table[ y=time, col sep=comma, x=baseline] {results/Exp1_serialize_cpu_io_bar.dat};
	   \addlegendentry{#7};
   };  
   
   
   \newcommand{\addDiagramExpOne}[1]{		
	% \addplot[draw=black,line width=0.15pt, fill=color2, discard if single={aaaa}] 
	% table[ y=time, col sep=comma, x=baseline] {results/Exp2_read_memory.dat};

	\addplot [forget plot,draw=none] coordinates{({HandCodedCPP},-1) ({inPlaceCPP},-1) ({FlatBuffersCPP},-1) ({ProtoBufCPP},-1) ({BoostCPP},-1) ({BsonCPP},-1) ({MessagePackRust},-1) ({BincodeRust},-1) ({JsonRust},-1) ({FlexBuffersRust},-1) ({BsonRust},-1) ({Json+GzipJava},-1) ({JsonJava},-1) ({BsonJava},-1) ({DefaultJava},-1) ({ProtoBufJava},-1) ({Byte BufferJava},-1) ({FlatBuffersJava},-1) ({KryoJava},-1)};

 	  \addplot[draw=black,line width=0.15pt, fill=tug, discard if single={CPP}] 
		table[ y=time, col sep=comma, x=baseline] {results/Exp2_read_memory.dat};
		\addlegendentry{C++};

		\addplot[draw=black,line width=0.15pt, fill=orange, discard if single={Rust}] 
		table[ y=time, col sep=comma, x=baseline] {results/Exp2_read_memory.dat};	
		\addlegendentry{Rust};

		\addplot[draw=black,line width=0.15pt, fill=tugb, discard if single={Java}] 
		table[ y=time, col sep=comma, x=baseline] {results/Exp2_read_memory.dat};	
		\addlegendentry{Java};
   };    
   
   \pgfplotsset{
	   discard if single/.style n args={1}{
		   x filter/.code={
			   \edef\tempa{\thisrow{language}}
			   \edef\tempb{#1}
			   \ifx\tempa\tempb
			   \else
			   \def\pgfmathresult{inf}
			   \fi			
		   }
	   }
   };
  
	   \begin{axis}[
		   name=ax,
		   every major tick/.append style={ thick,major tick length=2.5pt, gray},
			axis line style={gray},
		    ybar,        
		   %ymode=log,
		   ymin=0,
		   y tick label style={/pgf/number format/1000 sep={}},
		   x tick label style={/pgf/number format/1000 sep={}},
		   ytick={0,4,8,12,16,20,24,28,32},
		   yticklabels={0,4,8,12,16,20,24,28,32},
		   enlarge y limits={0.5,upper},
		   enlarge x limits=0.03,
		   ylabel={Used Memory[GB]},		   
		   ytick align=outside,
		   xtick align=outside,
		   xtick pos=left,
		   ytick pos=left,
		   yticklabel style = {font=\scriptsize},
		   ylabel style = {font=\scriptsize, yshift=-7pt},
		   xticklabel style = {font=\scriptsize, rotate=50, anchor=east, xshift=0pt},
		   xtick=data,
		   height=0.7\columnwidth,  
		   bar width=7pt,
		   width=1\columnwidth,   
		   ymajorgrids=true,
		   grid style=dotted,   
		   minor grid style={gray!50}, 
		   every axis plot/.append style={
				%ybar,
				%bar width=.2,
				bar shift=0pt,
				fill
        }, 
		   symbolic x coords={
				HandCodedCPP,
		   		inPlaceCPP,
		    	FlatBuffersCPP,
				ProtoBufCPP,
				BoostCPP,
				BsonCPP,
				MessagePackRust,
				BincodeRust,
				JsonRust,
				FlexBuffersRust,
				BsonRust,
				Json+GzipJava,
				JsonJava,
				BsonJava,
				DefaultJava,
				ProtoBufJava,
				Byte BufferJava,
				FlatBuffersJava,
				KryoJava},   
		   xticklabels={
				HandCoded,
				inPlace,
				 FlatBuffers,
				  ProtoBuf, 
				  Boost, 
				  Bson, 
				  MessagePack, 
				  Bincode, 
				  Json, 
				  FlexBuffers, 
				  Bson, 
				  Json+Gzip, 
				  Json, 
				  Bson, 
				  Default, 
				  ProtoBuf, 
				  Byte Buffer, 
				  FlatBuffers, 
				  Kryo}, 
			point meta=rawy,
			nodes near coords,
			nodes near coords align={vertical},
			every node near coord/.append style={
				font=\scriptsize,
				rotate=90,
				anchor=west,	
				/pgf/number format/.cd,
				/pgf/number format/precision=2,	
				},
				legend image code/.code={\draw [#1] (0cm,-0.1cm) rectangle (0.20cm,0.22cm); },	
				legend style = {
					every column/.append style   = {column sep=0.5cm, text width=30pt, draw=black},
					font=\scriptsize,
					%anchor=black,
					draw=none,
					legend columns=-1,
					anchor=north,
					legend cell align=left
				},
				   legend pos = {north west},	
		   ]  	  
		   \addDiagramExpOne{TRUE};			
	   \end{axis}		   
\end{tikzpicture}showvalu
%         \caption{Experiment2: Memory Usage in Reading of 4M Tweet Objects}
% \end{figure}

%  \tikzsetnextfilename{Exp3_seq_read_cpu_io}
%     \begin{figure*}[h]
%         \centering
%         \makeatletter
\newcommand\resetstackedplots{
	\makeatletter
	\pgfplots@stacked@isfirstplottrue
	\makeatother
	\addplot [forget plot,draw=none] coordinates{({HandcodedCPP},1) ({inPlaceCPP},1) ({FlatBufCPP},1) ({ProtoBufCPP},1) ({BoostCPP},1) ({BsonCPP},1) ({MessagePackRust},1) ({BincodeRust},1) ({JsonRust},1) ({FlexBufRust},1) ({BsonRust},1) ({Json+GzipJava},1) ({JsonJava},1) ({BsonJava},1) ({DefaultJava},1) ({ProtoBufJava},1) ({ByteBufferJava},1) ({FlatBufJava},1) ({KryoJava},1)};
}
\makeatother


\begin{tikzpicture}
	\newcommand{\myaddplot}[8]{
	   \addplot[xshift=#2,draw=#5,line width=0.15pt, fill=#3, discard if single={#1}{#4}{#8}, postaction={pattern=#6,pattern color=#5}] 
	   table[ y=time, col sep=comma, x=baseline] {results/Exp3_seq_read_cpu_io.dat};
	   \label{#4#1#8}
   };  

   \newcommand{\myaddplotnan}[9]{
	   \addplot[xshift=#2,draw=#9,line width=0.15pt, fill=#3, discard if single={#1}{#4}{#8}, %postaction={pattern=#6,pattern color=#5},
	   visualization depends on={value \thisrow{baseline} \as \xbaseline},
	   every node near coord/.append style={
		   check for zeronan/.code={
			   \pgfkeys{/pgf/fpu=true}
			   %\showvalue{\xbaseline}{HandcodedCPP}{inPlaceCPP}{FlatBufCPP}{MessagePackRust}{BincodeRust}{JsonRust}{FlexBufRust}{FlatBufJava}
			   \pgfkeys{/pgf/fpu=false}
		   },
		   check for zeronan,
	   }  
	   ] 
	   table[ y=time, col sep=comma, x=baseline] {results/Exp3_seq_read_cpu_io.dat};	  
	   \label{#4#1#8}
   };  
   
   \newcommand{\myaddplots}[6]{	   
	\addplot[xshift=#2,draw=#5,line width=0.15pt, fill=#3, discard if single={#1}{#4}{#6},
		forget plot,
		nodes near coords custom=1,
		every node near coord/.append style={
			black,			
			xshift=-5pt,
			anchor=west
		}
	] 
	table[ y=time, col sep=comma, x=baseline] {results/Exp3_seq_read_cpu_io.dat};	
   };   
   
   \newcommand{\addDiagramExpOne}[3]{	
		\resetstackedplots	
		\myaddplot{True}{-5}{#2}{IO}{#2}{none}{I/O Time(taskset True)}{#1};
  		\myaddplotnan{True}{-5}{#2!40}{CPU}{#2!40}{north east lines}{CPU Time(taskset True)}{#1}{#2};
		\myaddplots{True}{-5}{white}{Total}{none}{#1};
	    \resetstackedplots	 
		\myaddplot{False}{5}{#3}{IO}{#3}{none}{I/O Time(taskset False)}{#1};  
	    \myaddplotnan{False}{5}{#3!40}{CPU}{#3!40}{north west lines}{CPU Time(taskset False)}{#1}{#3};
		\myaddplots{False}{5}{white}{Total}{none}{#1};    
   };     

%baseline,language,taskset,execution,platform,seq_rand,nrow,time
   \pgfplotsset{
	   discard if single/.style n args={3}{
		   x filter/.code={
			   \edef\tempa{\thisrow{taskset}}
			   \edef\tempb{#1}
			   \ifx\tempa\tempb
					   \edef\tempe{\thisrow{execution}}
					   \edef\tempf{#2}
					   \ifx\tempe\tempf	
					   	%%%%%%%%%%%%%%%%%%%
						   \edef\tempg{\thisrow{language}}
						   \edef\temph{#3}
						   \ifx\tempg\temph	
						   		%%%%%%%%%%%%%%%
								   \edef\tempi{\thisrow{seq_rand}}
								   \edef\tempj{Sequential}
								   \ifx\tempi\tempj	
								   		\edef\tempk{\thisrow{platform}}
								   		\edef\templ{Single}
								   		\ifx\tempk\templ	
										   \edef\tempm{\thisrow{nrow}}
								   			\edef\tempn{10000000}
								   			\ifx\tempm\tempn	
											\else
								   			\def\pgfmathresult{inf}
								   			\fi
										\else
								   		\def\pgfmathresult{inf}
								   		\fi
								   \else
								   \def\pgfmathresult{inf}
								   \fi      
								%%%%%%%%%%%%%%%	
						   \else
						   \def\pgfmathresult{inf}
						   \fi      
						%%%%%%%%%%%%%%%%%%%	
					   \else
					   \def\pgfmathresult{inf}
					   \fi      
			   \else
			   \def\pgfmathresult{inf}
			   \fi			
		   }
	   },
	   nodes near coords custom/.style={
		large value/.style={                    
			rotate=90,
			anchor=east,			
		},
		small value/.style={
			rotate=90,
			anchor=east,
		},
		every node near coord/.style={
		  font=\scriptsize,
		  inner sep=0.5mm,
		  /pgf/number format/1000 sep={},
		  check for zero/.code={%
			\pgfkeys{/pgf/fpu=true}
			\pgfmathparse{\pgfplotspointmeta-1}
			\pgfmathfloatifflags{\pgfmathresult}{0}{
				\pgfkeys{/tikz/coordinate}
			}{
				\begingroup                      
					\pgfkeys{/pgf/fpu}%
					\pgfmathparse{\pgfplotspointmeta<#1}%
					\global\let\result=\pgfmathresult
				\endgroup
				\pgfmathfloatcreate{1}{1.0}{0}%
				\let\ONE=\pgfmathresult
				\ifx\result\ONE
					% AH : our condition 'y < #1' is met.
					\pgfkeysalso{/pgfplots/small value}%
				\else
					% ok, proceed as usual.
					\pgfkeysalso{/pgfplots/large value}%
				\fi
			}
			\pgfkeys{/pgf/fpu=false}
		  },
		  check for zero,
		},
	},   
   };
   
   
	   \begin{axis}[
			name=ax,
		   %xtick style={draw=none},		
			every major tick/.append style={thick,major tick length=3.5pt, gray},
			axis line style={gray},
		    ybar stacked,        
		   ymode=log,
		   %ymin=5,
		   %y tick label style={/pgf/number format/1000 sep={}},
		   %x tick label style={/pgf/number format/1000 sep={}},
		   %scaled y ticks=False,
		   enlarge y limits={0.42,upper},
		   enlarge x limits=0.03,
		   ylabel={Execution Time[s] - Log},		   
		   ytick align=inside,
		   xtick align=outside,			
		   xtick pos=left,
		   ytick pos=left,
		   yticklabel style = {font=\scriptsize},
		   ylabel style = {font=\scriptsize, yshift=-12pt},
		   xticklabel style = {font=\scriptsize},
		   xtick=data,
		   height=0.65\columnwidth,  
		   bar width=10pt,
		   width=1.1\textwidth,   
		   ymajorgrids=true,
		   grid style=dotted,   
		   minor grid style={gray!50},  
		   symbolic x coords={ HandcodedCPP, inPlaceCPP, FlatBufCPP, ProtoBufCPP, BoostCPP, BsonCPP, MessagePackRust, BincodeRust, JsonRust, FlexBufRust, BsonRust, Json+GzipJava, JsonJava, BsonJava, DefaultJava, ProtoBufJava, ByteBufferJava, FlatBufJava, KryoJava},   
		   xticklabels={ Handcoded,, FlatBuf,, Boost,, MessagePack,, Json,, Bson,, Json,, Default,, Byte Buffer,, Kryo},  
		   %-----------------		   
		   extra x ticks={inPlaceCPP,ProtoBufCPP,BsonCPP,BincodeRust,FlexBufRust,Json+GzipJava,BsonJava, ProtoBufJava,FlatBufJava},
		   extra x tick labels={inPlace,ProtoBuf,Bson,Bincode,FlexBuf,Json+Gzip,Bson,ProtoBuf,FlatBuf},
		   every extra x tick/.style={major tick length=15pt, xtick align=outside},
		   %----------------		   
		   point meta=rawy,            
           nodes near coords={\pgfmathprintnumber[precision=1]{\pgfplotspointmeta}},
           nodes near coords custom={101}, 
		   legend image code/.code={\draw [#1] (0cm,-0.1cm) rectangle (0.15cm,0.20cm); },					   
		   ]  	  
		   \addDiagramExpOne{CPP}{tug}{orange};		
		   \addDiagramExpOne{Rust}{tugb}{color7};	
		   \addDiagramExpOne{Java}{color5}{color2};	
		%    \node [draw=none,inner sep=0, font=\scriptsize, anchor=west](leg1) at (rel axis cs: 0.01,0.92) {\shortstack[l]{
        %     \ref{IOTrueCPP} , \ref{IOTrueRust} , \ref{IOTrueJava} I/O Time(taskset True) \quad
        %     \ref{IOFalseCPP} , \ref{IOFalseRust} , \ref{IOFalseJava} I/O Time(taskset False) \quad
		% 	\ref{CPUTrueCPP} , \ref{CPUTrueRust} , \ref{CPUTrueJava} CPU Time(taskset True) \quad			
        %     \ref{CPUFalseCPP} , \ref{CPUFalseRust} , \ref{CPUFalseJava} CPU Time(taskset False)
        % }};		
		% \node [draw=gray!50,inner sep=0.5mm, font=\scriptsize, anchor=west] at (rel axis cs: 0.01,0.70) {\shortstack[l]{
        %     \ref{IOTrueCPP} , \ref{IOFalseCPP} , \ref{CPUTrueCPP} , \ref{CPUFalseCPP}  C++ \\
        %     \ref{IOTrueRust} , \ref{IOFalseRust} , \ref{CPUTrueRust} , \ref{CPUFalseRust}  Rust \quad \\
		% 	\ref{IOTrueJava} , \ref{IOFalseJava} , \ref{CPUTrueJava} , \ref{CPUFalseJava}  Java
        % }};		
	   \end{axis}		   
	 \end{tikzpicture}
%         \caption{Experiment3: Total and IO Time of Sequential Read of 4 Million Tweet objects}
% \end{figure*}


\maketitle
\IEEEdisplaynontitleabstractindextext
\IEEEpeerreviewmaketitle
\bibliographystyle{IEEEtran}
\bibliography{References}

\end{document}